



## Cointegración




### Cointegración como equilibrio de largo plazo

- Considere el modelo monetario
\begin{equation*}
\notation{m_t}{oferta} = \notation{\beta_0 + \beta_1p_t+\beta_2y_t+\beta_3r_t}{demanda} + \notation{\epsilon_t}{brecha}
\end{equation*}
- Para que la noción de equilibrio tenga sentido, la brecha $\epsilon_{t}$ debe ser estacionaria, es decir, I(0)
- Esto a pesar de que $m_t$, $p_t$, $y_t$ y $r_t$ sean I(1).
- Esto implica que la combinación lineal de variables I(1)
\begin{equation*}
\MAT{1& -\beta_1 & -\beta_2 & -\beta_3}\MAT{m_t \\ p_t \\ y_t \\ r_t} = \beta_0+\epsilon_{t}
\end{equation*}
resulta en un proceso I(0)
- Decimos que $m_t$, $p_t$, $y_t$ y $r_t$ están cointegradas, con vector de cointegración $\MAT{1& -\beta_1 & -\beta_2 & -\beta_3}$.
- \alert{Teorías de equilibrio con variables no estacionarias requieren la existencia de una combinación lineal de las variables que sea estacionaria}
- Otros ejemplos:

    - Teoría de la función de consumo:
    \begin{equation*}
    c_t = \beta y_t^p + c_t^t \quad\Rightarrow \MAT{1 &-\beta}\MAT{c_t \\ y_t^p} = c_t^t \quad\text{es estacionario}
    \end{equation*}
    - Teoría de la paridad del poder de compra:
    \begin{equation*}
    e_t + p_t^* - p_t = \MAT{1 & 1 & -1}\MAT{e_t \\ p_t^*\\ p_t} \quad\text{es estacionario}
    \end{equation*}




### Regresión espuria

- Considere dos caminatas aleatorias independientes
\begin{align*}
y_t &= y_{t-1} + u_t \qquad u_t \text{ ruido blanco}\\
x_t &= x_{t-1} + v_t \qquad v_t \text{ ruido blanco}
\end{align*}
- Como $y_t$ es independiente de $x_t$, uno esperaría que en la regresión
\begin{equation*}
y_t = \beta_0 + \beta_1 x_t + \epsilon_t
\end{equation*}
el $R^2$ y el $\beta_1$ tendieran a cero.
- Pero este no es el caso. Con series no estacionarias, \alert{la correlación espuria puede persistir aún en muestras grandes}.




\begin{EXAMPLE}[datos-CRI-MLT.pickle \\ MacKinnon 2010 Critical values for cointegration tests.xlsx\\ \midrule Relación espuria.ipynb\\ mackinnon.py]{Regresión espuria}

###<beamer>
\includegraphics[width=\linewidth]{Malta-map.png}



###[fragile]

Regresión con series I(1):
\begin{equation*}
\log(\text{GDP.COSTA.RICA}_t) = \beta_0 + \beta_1\log(\text{GDP.MALTA}_t) + \epsilon_t
\end{equation*}

\only<presentation>{\tiny}
\begin{verbatim}
                        OLS Regression Results
==============================================================================
Dep. Variable:                    CRI   R-squared:                       0.806
Model:                            OLS   Adj. R-squared:                  0.802
Method:                 Least Squares   F-statistic:                     195.0
Date:                Thu, 02 Jan 2020   Prob (F-statistic):           2.40e-18
Time:                        14:07:58   Log-Likelihood:                 34.197
No. Observations:                  49   AIC:                            -64.39
Df Residuals:                      47   BIC:                            -60.61
Df Model:                           1
Covariance Type:            nonrobust
==============================================================================
             coef    std err          t      P>|t|      [0.025      0.975]
------------------------------------------------------------------------------
Intercept      4.4814      0.300     14.916      0.000       3.877       5.086
MLT            0.4432      0.032     13.966      0.000       0.379       0.507
==============================================================================
Omnibus:                       57.494   Durbin-Watson:                   0.064
Prob(Omnibus):                  0.000   Jarque-Bera (JB):                5.218
Skew:                           0.156   Prob(JB):                       0.0736
Kurtosis:                       1.432   Cond. No.                         164.
==============================================================================
\end{verbatim}


###[fragile]
Regresión con series I(0):
\begin{equation*}
\Delta\log(\text{GDP.COSTA.RICA}_t) = \beta_0 + \beta_1\Delta\log(\text{GDP.MALTA}_t) + \epsilon_t
\end{equation*}

\only<presentation>{\tiny}
\begin{verbatim}
                        OLS Regression Results
==============================================================================
Dep. Variable:                    CRI   R-squared:                       0.032
Model:                            OLS   Adj. R-squared:                  0.011
Method:                 Least Squares   F-statistic:                     1.537
Date:                Thu, 02 Jan 2020   Prob (F-statistic):              0.221
Time:                        14:09:30   Log-Likelihood:                 102.51
No. Observations:                  48   AIC:                            -201.0
Df Residuals:                      46   BIC:                            -197.3
Df Model:                           1
Covariance Type:            nonrobust
==============================================================================
             coef    std err          t      P>|t|      [0.025      0.975]
------------------------------------------------------------------------------
Intercept      0.0143      0.006      2.269      0.028       0.002       0.027
MLT            0.1378      0.111      1.240      0.221      -0.086       0.362
==============================================================================
Omnibus:                       27.205   Durbin-Watson:                   1.244
Prob(Omnibus):                  0.000   Jarque-Bera (JB):               58.241
Skew:                          -1.586   Prob(JB):                     2.26e-13
Kurtosis:                       7.365   Cond. No.                         26.4
==============================================================================
\end{verbatim}

\end{EXAMPLE}



### Definición de cointegración \parencite{EngleGranger:1987}
Se dice que los componentes del vector $x_t = \left(x_{1t},x_{2t},\dots,x_{nt}\right)'$ están \alert{cointegrados} de orden $(d, b)$, denotado por $x_t\sim CI(d,b)$, si
\begin{enumerate}
- Todos los componentes de $x_t$ son integrados de orden $d$.
- Existe al menos un vector $\beta = \left(\beta_1,\beta_2,\dots,\beta_n\right)$ tal que la combinación lineal $\beta x_t = \beta_1x_{1t} + \beta_2x_{2t} + \dots +\beta_nx_{nt}$ es integrada de orden $(d-b)$, donde $b>0$.

\vspace{2em}
A $\beta$ se le llama \alert{vector de cointegración}
\end{enumerate}


### Algunas observaciones acerca de la definición de cointegración
\begin{enumerate}
- Cointegración se refiere a combinaciones \emph{lineales} de variables no estacionarias
- Si existe, el vector de cointegración no es único
- Cointegración se refiere a variables del mismo orden;aunque es posible encontrar relaciones de equilibrio entre variables de distinto orden
- Pueden existir varios vectores de cointegración independientes para un conjunto de variables $x_t$
- En la mayor parte de la literatura se entiende cointegración como el caso $CI(1,1)$.

\end{enumerate}





### Pruebas de cointegración: Engle-Granger
Una receta para determinar si las series están cointegradas:

\begin{tabu} to \textwidth {X[2cm]X[cm]}
\begin{description}
-[Ingredientes:] series de tiempo, software econométrico
-[Paso 1:] Determinar orden de integración de las series
-[Paso 2:] Estimar la relación de equilibrio de largo plazo
-[Paso 3:] Estimar el modelo de corrección de errores
-[Paso 4:] Evaluar si el modelo es adecuado
\end{description}
&
\includegraphics<beamer>[width=\linewidth]{recipe-cooking}
\end{tabu}









### Pruebas de cointegración de Engle-Granger
En la prueba (aumentada) de Engle y Granger,
\begin{align}
\Delta\hat{\epsilon}_t &= \alert{\gamma}\hat{\epsilon}_{t-1}  +\sum_{i=1}^{n}a_i\Delta\hat{\epsilon}_{t-i} + \varepsilon_t \tag{nc} \\
\Delta\hat{\epsilon}_t &= \alpha_0 + \alert{\gamma}\hat{\epsilon}_{t-1}  +\sum_{i=1}^{n}a_i\Delta\hat{\epsilon}_{t-i} + \varepsilon_t \tag{c}  \\
\Delta\hat{\epsilon}_t &= \alpha_0 + \alpha_1 t + \alert{\gamma} \hat{\epsilon}_{t-1}  +\sum_{i=1}^{n}a_i\Delta\hat{\epsilon}_{t-i} + \varepsilon_t \tag{ct}  \\
\Delta\hat{\epsilon}_t &= \alpha_0 + \alpha_1 t + \alpha_2 t^2 +\alert{\gamma} \hat{\epsilon}_{t-1}  +\sum_{i=1}^{n}a_i\Delta\hat{\epsilon}_{t-i} + \varepsilon_t \tag{ctt}
\end{align}
si $\gamma=0$ los residuos $\hat{\epsilon}_t$ presentan raíz unitaria, y por ello las series no estaría cointegradas.





### Los valores críticos de MacKinnon

- Para probar la hipótesis nula $H_0: \gamma=0$ contra la alternativa $H_1: \gamma<0$, se utiliza el estadístico $t_\gamma$.
- No obstante, \alert{$t_\gamma$ no tiene la distribución $t-$student, ni siquiera asintóticamente}.
- Dado que no es posible derivar la distribución de $t_\gamma$ analíticamente, es necesario aproximarla con simulaciones de Monte Carlo.
- A partir de tales simulaciones, \textcite{MacKinnon:2010} presenta valores críticos, que dependen de
    - la especificación determinística (nc, c, ct, ctt),
    - del número de series en el vector de cointegración
    - y del tamaño de muestra $T$.
- Los valores se obtienen evaluando un polinomio en  $\frac{1}{T}$
\begin{equation*}
C(p) = \beta_\infty + \beta_1 T^{-1} + \beta_2 T^{-2} + \beta_3 T^{-3}
\end{equation*}
- Por ejemplo, para probar la cointegración de $N=3$ variables, con constante y tendencia (ct), la tabla es

\begin{center}
\begin{scriptsize}
\taburulecolor{orange}
\begin{tabu} to 0.7\textwidth {X[c]*4{X[r]}}
\toprule \rowfont[c]{\bfseries}
Nivel & $\beta_{\infty}$ & $\beta_1$ & $\beta_2$ & $\beta_3$ \\ \midrule
1 \%  &     -4.663     & -18.769  &  -49.793  &  104.244  \\
5 \%  &     -4.119     & -11.892  &  -19.031  &  77.332   \\
10\%  &     -3.835     &  -9.072  &   -8.504  &  35.403   \\ \bottomrule
\end{tabu}
\end{scriptsize}
\end{center}
- Así, si tenemos 50 observaciones:
\begin{equation*}
C(5\%) = -4.119  -\tfrac{11.892}{50} -\tfrac{19.031}{50^2} + \tfrac{77.332}{50^3} = -4.3637
\end{equation*}





\begin{EXAMPLE}{Valores críticos de Mackinnon}
###

- En el cuaderno de Jupyter \texttt{Mackinnon valores críticos para test de cointegración} se presentan más ejemplos.
- Se muestra cómo los valores críticos cambian con el tamaño de muestra, el número de series que conforman el vector, y la especificación de los componentes determinísticas de las series.


\end{EXAMPLE}




\begin{SIDENOTE}{Factorización de rango}
###[label=appendix-factorizar-rango]
Suponga que $A$ es una matriz $n\times n$ con rango $r<n$. Existen las matrices $X$ y $Y$ de dimensión $r\times n$ tal que:
\begin{equation*}
A = X'Y
\end{equation*}

\end{SIDENOTE}





## Vector de Corrección de Errores (VECM)



### De VAR a VECM
\vspace*{-1.5em}
\begin{align*}
y_t &= \Phi_1y_{t-1} + \Phi_2y_{t-2} + \Phi_3y_{t-3} + \epsilon_t \\
\uncover<2->{\Delta y_t &= \left(\Phi_1 - I\right)y_{t-1} + \Phi_2y_{t-2} + \Phi_3y_{t-3} + \epsilon_t \\}
\uncover<3->{&= \left(\Phi_1 - I\right)y_{t-1} + \left(\Phi_2 \alert{+\Phi_3}\right)y_{t-2} + \Phi_3\left(y_{t-3}\alert{- y_{t-2}}\right) + \epsilon_t \\}
\uncover<4->{&= \begin{multlined}
\left(\Phi_1\alert{+\Phi_2 +\Phi_3} - I\right)y_{t-1} + \left(\Phi_2 +\Phi_3\right)\left(y_{t-2}\alert{- y_{t-1}}\right) + \\ \dots + \Phi_3\left(y_{t-3}- y_{t-2}\right) + \epsilon_t
\end{multlined} } \\
\uncover<5->{&= \left(\Phi_1+\Phi_2 +\Phi_3 - I\right)y_{t-1} - \left(\Phi_2 +\Phi_3\right)\Delta y_{t-1}  - \Phi_3\Delta y_{t-2} + \epsilon_t \\}
\uncover<6->{&= \Pi y_{t-1} +\Gamma_1\Delta y_{t-1}  +\Gamma_2\Delta y_{t-2} + \epsilon_t }
\end{align*}

\uncover<7->{\small Si hay $0<r<n$ vectores de cointegración, entonces $\Pi$ puede ser descompuesta como el producto de los vectores de cointegración $\beta$ y los coeficientes de corrección de errores $\alpha$:} %\hyperlink{appendix-factorizar-rango}{\beamerbutton{rango}}

\uncover<8->{
\begin{RESALTADO}[RoyalBlue4][1.0]{VEC}
\begin{equation*}
\Delta y_t = \alpha\beta' y_{t-1} + \Gamma_1\Delta y_{t-1} +\dots + \Gamma_{p-1}\Delta y_{t-p+1} + \epsilon_t
\end{equation*}
\end{RESALTADO}
}


\begin{EXAMPLE}{Inflación y depreciación en un modelo VEC}

- Suponga que en el largo plazo se cumple la PPP: $p_t = e_t + p_t^* + \text{error}_t$
y que las tres variables son I(1) y relacionadas como un VAR(2)
- La representación VECM es
\small
\begin{align*}
\Delta p_t^* & = \alpha_1\left(p_t - e_t - p_t^*\right) + \gamma_{11}\Delta p_{t-1}^* + \gamma_{12}\Delta e_{t-1} + \gamma_{13}\Delta p_{t-1} + \epsilon_{1t}\\
\Delta e_t & = \alpha_2\left(p_t - e_t - p_t^*\right) + \gamma_{21}\Delta p_{t-1}^* + \gamma_{22}\Delta e_{t-1} + \gamma_{23}\Delta p_{t-1} + \epsilon_{2t}\\
\Delta p_t & = \alpha_3\left(p_t - e_t - p_t^*\right) + \gamma_{31}\Delta p_{t-1}^* + \gamma_{32}\Delta e_{t-1} + \gamma_{33}\Delta p_{t-1} + \epsilon_{3t}
\end{align*}
- o bien
\begin{equation*}
\MAT{\Delta p_t^* \\ \Delta e_t \\ \Delta p_t} =
\MAT{\alpha_1 \\ \alpha_2 \\ \alpha_3}\MAT{-1 & -1 & 1}
\MAT{p_{t-1}^* \\ e_{t-1}\\ p_{t-1}} +
\Gamma_1 \MAT{\Delta p_{t-1}^* \\ \Delta e_{t-1}\\ \Delta p_{t-1}} +
\epsilon_t
\end{equation*}
- Este modelo explica la inflación internacional, la depreciación, y la inflación doméstica en función de sus propios rezagos y la desviación de los precios y tipo de cambio respecto a su equilibrio de largo plazo.


\end{EXAMPLE}




### Pruebas de cointegración: Johansen

- La prueba de Johansen puede verse como una generalización multivariada de la prueba aumentada de Dickey-Fuller
- La prueba y estrategia de estimación permiten estimar \emph{todos} los vectores de cointegración
- Similar a la prueba ADF, la existencia de raíces unitarias implican que la teoría asintótica estándar no es apropiada.



### Raíces unitarias: modelo univariado versus multivariado

- Comparemos la prueba ADF con el VECM
\begin{small}
\begin{align*}
\Delta y_t &=  \pi y_{t-1} +\gamma_1\Delta y_{t-1} + \dots +\gamma_p\Delta y_{t-p} + \epsilon_t \tag{univariado} \\
\Delta y_t &=  \Pi y_{t-1} +\Gamma_1\Delta y_{t-1} + \dots +\Gamma_p\Delta y_{t-p} + \epsilon_t \tag{multivariado}
\end{align*}
\end{small}
- En la prueba ADF, probamos si $y_t$ tiene raíz unitaria con $\text{H}_0: \pi=0$
- En el caso multivariado, Johansen determina si las series están cointegradas a partir del \alert{rango} de $\Pi$



### Rango de una matriz vectores de cointegración
Posibles casos del rango:
\begin{description}
-[0:] implica $\Pi=0$, todas las series son I(1) pero no están cointegradas. VAR en diferencias
-[$0<r<N$:] hay $r$ vectores de cointegración, y escribimos $\Pi=\alpha\beta'$. VECM
-[N:] \alert{cualquier} combinación lineal es estacionaria, lo que implica que las series originales eran estacionarias.
\end{description}


### Usando los eigenvalores para determinar el rango

- El rango de una matriz es igual al número de sus eigenvalores distintos de cero.
- Por ello, las pruebas de Johansen están basadas en los eigenvalores de una matriz $\Pi^*$ semidefinida positiva, \alert{derivada a partir de $\Pi$}.
- Suponga que obtenemos $\Pi^*$ y ordenamos sus eigenvalores de manera tal que
\begin{equation*}
\lambda_1 \geq \lambda_2 \geq \dots \geq \lambda_N \geq 0
\end{equation*}



### Prueba de la traza y del máximo eigenvalor
\begin{RESALTADO}{Prueba de la traza}
\begin{equation*}
\lambda_{\text{traza}}(r) = - T\sum_{i=r+1}^{N}\ln\left(1 - \hat{\lambda}_i\right)
\end{equation*}
\end{RESALTADO}



\begin{RESALTADO}{Prueba del máximo eigenvalor}
\begin{equation*}
\lambda_{\max}(r, r+1) = - T\ln\left(1 - \hat{\lambda}_{r+1}\right)
\end{equation*}
\end{RESALTADO}



Note que $\lambda_i \geq 0 \Rightarrow \ln\left(1-\lambda_i\right) \leq 0$. Ambos estadísticos son no-negativos
Valores grandes de los estadísticos apuntan a que los eigenvalores son positivos, implicando la existencia de cointegración.




\begin{EXAMPLE}{Pruebas de Johansen}
- Johansen y Juselius (1990) analizan la cointegración de $\MAT{m2_t & y_t & i^d_t & i^b_t}$, con datos trimestrales de Dinamarca para el período 1974:1 a 1987:3 (T=53).

\begin{center}
\small  %fuente: Enders (2014) Applied Econometric Time Series, p379
\taburulecolor{orange}
\begin{tabu} to 0.7\textwidth {X[1.2cl]X[cm]X[cm]X[cm]}
\toprule
$H_0$ & $\hat{\lambda}_i$ & $\lambda_{\max}$ & $\lambda_{\text{traza}}$  \\
\midrule
$r=0$ & 0.4332 & 30.09 & 49.14 \\
$r=1$ & 0.1776 & 10.36 & 19.05 \\
$r=2$ & 0.1128 & 6.34 & 8.69 \\
$r=3$ & 0.0434 & 2.35 & 2.35 \\
\bottomrule
\end{tabu}
\end{center}

\end{EXAMPLE}




\makeReferencesFrame{Enders:2014}
